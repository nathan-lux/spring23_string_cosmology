\documentclass[preprint]{ptephy_v1}
\usepackage{xcolor} 
\usepackage{amsmath}
\usepackage{amssymb}
\usepackage{amsfonts}
\usepackage{array}
\usepackage{graphicx}% use this package if an eps figure is included.
\usepackage{mathrsfs}
\usepackage{multirow}
\newcounter{questioncounter}
\newcounter{equestioncounter}
\setlength\parskip{5pt} \setlength\parindent{0in}
\newcommand{\bea}{\begin{eqnarray*}}
\newcommand{\eea}{\end{eqnarray*}}
\newcommand{\beao}{\begin{eqnarray}}
\newcommand{\eeao}{\end{eqnarray}}
\newcommand{\no}{\noindent}
\makeatletter
\newcommand \Dotfill {\leavevmode \cleaders \hb@xt@ .5em{\hss .\hss }\hfill \kern \z@}
\makeatother

\begin{document}
\title{Varying Actions in String Frame and Einstein Frame to Compare EOM and
Phase Space Diagrams}
\author{Nathan Burwig\\
\small Arizona State University Department of Cosmology, Particle, and Astrophysics}

\begin{abstract}
    In this section of the paper we will aim to recover the string frame
    equations of motion from Scott Watson's papers on string cosmology.
    \vspace{-3pt}

    \Dotfill
\end{abstract}

\maketitle
\newpage
\section{Varying The String Frame Action}
    In this section we will attempt to derive the string frame equations of
    motion from a general action and in doing so we will derive the Hamiltonian
    constraint (ie one of the Friendmann Equations). We will use these
    equations of motion to formulate a phase space for the dilaton and string
    frame Hubble parameter, wherupon we will consider the same action in the
    Einstein frame. A phase space in the Einstein frame for the actual dilaton
    and the actual hubble parameter will be produced, and interpretations on
    early universe (ie Hagerdorn and possibly post hagerdorn) phase dynamics
    will be offered.

    We will start with the string frame action in it's full form before
    considering reductions and simplifications of the action.

    \subsection{The Full Action}
    The typical action considered in string cosmologies (as is done in
    \cite{vafa} and \cite{kaloper}) is a general N-dimensional action which
    allows us to include any general number of extensive dimensions (typically
    here we will consider d=3).
    \begin{equation}
        S=\frac{1}{16\pi G_n} \int d^nx\sqrt{-g}e^{-2\phi}\left(
        R+4g^{\mu\nu}\nabla_{\mu}\phi\nabla_{\nu}\phi-\frac{1}{12}
        H_{\mu\nu\lambda}H^{\mu\nu\lambda} \right) + \int d^nx \mathcal{L}_m
    \end{equation}
    As is typical of the literature, the three form Einstein-Maxwell tensor
    $H$ will not be important for this discussion and thus will be ignored from
    the action. Considerations of the matter Lagrangian will in fact be
    important, however alternatively it may be excluded from explicit
    calculation in favor of simply asserting a form of the energy momentum
    tensor which will also be explored to some extent.

    If we ignore the three form tensor $H$, then we are left with the much
    simpler action
    \begin{equation}
        S=\frac{1}{2\kappa_N} \int d^{N+1}x\;\sqrt{-g}\;e^{-\phi_s}\left(
        R+(\partial \phi_s)^2 - \mathcal{L}_m \right)
    \end{equation}
    Which we can vary with respect to the string frame metric $g$ in a typical
    fashion. The nontrivial coupling with the scalar field $\phi_s$, however it
    is not disimilar to a Brans-Dicke action and thus it's variation is well
    known to some extent.
    
    We consider the action in three parts, the variation of the Ricci scalar
    which we know results in the typical field equations, then the variation of
    the scalar field derivatives and matter Lagrangian with the determinant of
    the metric. Considerations of the shape of the metric will be important, as
    well as the specific form of the Ricci scalar as we can simply take the
    timelike component of the action to determine the equations of motion. The
    pure timelike component will result in the equivalent of the Friedmann
    equation in the string frame (ie the Hamiltonian constraint) while the
    equations of motion for $\phi_s$ and $\lambda_s$ (where $\lambda_s=\ln(a)$)
    will not generically depend on the variation of the action with respect to
    the metric and thus it doesn't matter if we consider the purely timelike
    term or the whole action.

    For what seems to be historical reasons it is typical to start with the
    anisotropic FLRW metric 
    \begin{equation}
        ds^2=-n(t)^2dt^2+\sum_{i=1}^N\;e^{2\lambda_s^{(i)}(t)}dx_i^2
    \end{equation}
    And then consider specializations to isometric metrics (ie all
    $\lambda_s^(i)$ in the metric are identical) and with
    $a_s^{(i)}=e^{\lambda_s^{(i)}(t)}$. Then we can consider a new form of the
    metric by determining the form of the Ricci scalar.

    %% \newpage
    %% \bibliographystyle{plain}
    %% \bibliography{mybib}
\end{document}
